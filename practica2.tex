\section{Práctica 2}

\subsection{}

\subsubsection{}

CPU: [0,2], [11,13], [21,22]
E/S: [3,10], [14,20]

\subsubsection{}

3 y 2 las de CPU, 7 las de E/S

\setcounter{subsection}{3}
\subsection{}

\subsubsection{}

No

\subsubsection{}

Si, que se este ejecutando una tarea de cierta prioridad, llegue una de menor
prioridad que no es atendida, y que a partir de ahí sigan llegando tareas de
mayor prioridad ocasionando que nunca se atienda la tarea de menor prioridad.

\subsubsection{}

Si, que se este ejecutando una tarea de cierta duración, llegue una de mayor
duración que no es atendida, y que a partir de ahí sigan llegando tareas de
menor duración ocasionando que nunca se atienda la tarea de mayor duración.

\subsubsection{}

No

\subsection{}

\subsubsection{}

El efecto es que ese proceso se ejecuta más

\subsubsection{}

Ventaja es que puede discriminarse el quantum de cada proceso, lo cual
permite darle importancia en cierto sentido.

La desventaja es que si se usan múltiples procesadores podría ocurrir que
mientras un procesador este ejecutando una tarea, otro caiga en el PCB
duplicado

\subsubsection{}

Puede utilizarse una lista con los PCB no duplicados y otra con las duraciones
de los quantums de cada proceso.

\setcounter{subsection}{6}
\subsection{}

\subsubsection{}

\begin{tabular}{c c c c c}
Tarea & $T_0$ & $T_1$ & $T_2$ & Promedio \\
Espera & 16 & 21 & 3 & 13.3 \\
Turnaround & 32 & 34 & 24 \\
\end{tabular}