\section{Práctica 4}

\subsection{}

Una dirección de memoria lógica es traducida a una de memoria lineal a
través de la segmentación, y ésta es traducida a una dirección de memoria
física mediante la paginación.

\subsection{}

Fragmentación interna es cuando los bloques son muy grandes para los datos
almacenados dentro de ellos, lo que provoca que se desperdicie espacio.
Fragmentación externa es cuando no hay suficiente memoria contigua para ser
asignada, por lo que queda inutilizable.

\subsection{}

Bloques
\begin{enumerate}
	\item 8MB 2MB
	\item 1MB
	\item 4MB 1MB
	\item 512KB 12KB
	\item 512KB
	\item 2MB
\end{enumerate}

Programas
\begin{enumerate}
	\item 500KB
	\item 6MB
	\item 3MB
	\item 20KB
	\item 4MB
\end{enumerate}

\subsubsection{}

\begin{itemize}
	\item Programa 1 en bloque 4
	\item Programa 2 en bloque 1
	\item Programa 3 en bloque 3
	\item Programa 4 en bloque 2
	\item Programa 5 en bloque 1
\end{itemize}

\subsection{}

Porque con una dirección lineal de $n$ cantidad de bits pueden direccionarse
$2^n$ páginas, por lo que una la tabla tiene $2^n$ entradas.

\setcounter{subsection}{6}
\subsection{}

65536 bytes divididos en páginas de 4096 bytes hacen 16 páginas en total. Se
necesitan 8 páginas para el texto, 5 para los datos y 4 para la pila, por lo
que no es posible ejecutarlo.

Si el tamaño de página fuera de 512 bytes, se tendrían direccionadas 128
páginas. El texto del programa requeriría 64, los datos 33 y la pila 31. De
esta forma sí podria ejecutarse.

\subsection{}

\subsubsection{}

400, porque se debe acceder primero a la entrada correspondiente de la tabla y
una segunda vez para la página.

\subsection{}

Ocurre cuando se intenta acceder a una página que ya no está en memoria. Lo
que el sistema hace en ese caso es guardar en disco una de las páginas ya
cargadas en memoria para hacer espacio a la página que se está intentando
acceder. Una vez hecho esto se lee la página de disco y se la carga en
memoria.

\setcounter{subsection}{10}
\subsection{}

\subsubsection{}

Cada pagina contiene 200 posiciones por lo que habrá 50 fallos de página

\subsubsection{}

Como la matriz está almacenada contiguamente por por fila, este tipo de
recorrido salta a través de las columnas, por lo que cada pagina solo
albergará 2 posiciones, por lo que habran 5000 fallos de página

\setcounter{subsection}{12}
\subsection{}

\subsubsection{}

Tiene sentido si lo que se hace es dejar un segmento especialmente para
atender las llamadas. Si fuese parte de la paginación, sería posible que la
página en donde está almacenado el código para la atención de llamadas se
encuentre fuera de memoria, por lo que se perdería tiempo en traerla de disco.

\subsubsection{}

Si, justamente para ahorrarse el tiempo de ir a buscarlas a memoria.