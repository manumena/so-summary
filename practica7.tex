\section{Práctica 7}

\subsection{}

\subsubsection{}
Se aplica la función de hash a la contraseña que ingresó el usuario y si el
resultado coincide con el valor almacenado entonces el ingreso es exitoso.

\subsubsection{}
La probabilidad de acertar con una contraseña distinta es de $\frac{1}{2^64}$.

\subsubsection{}
Cantidad de contraseñas a probar: $2^63$.

Cantidad de contraseñas por segundo: $10^9$.

Cantidad de segundo por año: $3.15 \times 10^7$.

Serían necesarios $\frac{2^63}{10^16 \times 3.15} ~= 293$ años.

\subsubsection{}
Cantidad de contraseñas a probar: $36^6$.

Cantidad de contraseñas por segundo: $10^9$.

Serían necesarios $\frac{36^6}{10^9} = 2.18$ segundos.

\subsection{}

\subsubsection{}
No es seguro ya que es vulnerable a un replay-attack. Consiste en interceptar
la información y retransmitirla, haciéndose pasar por otra persona.

\subsubsection{}
El atacante tendría el seed y el hash, lo que podría hacer es hashear todas
las contraseñas posibles con ese seed hasta que coincida con el hash robado.
Pero para eso necesitaría tener la función de hash.

\subsection{}

\subsubsection{}
El problema se introduce en la funcion \textit{gets}, puesto que esta no se
fija que la entrada tenga menor tamaño que la variable a donde se almacenará,
y como consecuencia el contenido pisa el resto de la pila de la función
pudiendo pisarse la dirección de retorno, desencadenando la ejecución de otro
programa almacenado en otra posición de memoria.

\subsubsection{}
Únicamente la variable nombre.

\subsubsection{}
La dirección de retorno de \textit{gets} si.

\subsubsection{}
No. Una vez que se llama a \textit{gets} el usuario ya tomó el control.

\subsection{}
Primero se apilan los parámetros de \textit{login}, luego la dirección de
retorno y finalmente las variables locales, que en este caso son un arreglo de
char de 32 posiciones y un credential que contiene otros 2 arreglos de char de
32 posiciones cada uno.

Las variables se apilan de manera tal que al leerlas o escribirlas se hace
descendiendo por la pila. El problema es que \textit{fgets} puede escribir
hasta la cantidad de caracteres que se pasa por parámetro, y en este caso es
errónea ya que debería ser \textit{sizeof(user.pass)} en lugar de
\textit{sizeof(user)}. Esto permite que puedan escribirse los siguientes 32
bytes de la pila, que en el caso del segundo \textit{fgets}, es el espacio
reservado para la contraseña real. Y asi puede sobreescribirse en la pila la
verdadera contraseña y engañar al programa.

\subsubsection{}
Los valores son \textit{admin} para el usuario, y
\textit{cualquiercontraseñad32caracterescualquiercontraseñad32caracteres} para
la contraseña.

\subsection{}

\subsubsection{}
El mecanismo que hace esto posible es el parámetro setUid a la hora de asignar
los permisos del ejecutable de este código. Se establece que este programa que
puede ser ejecutado por cualquier usuario, tiene permisos para abrir o
ejecutar otros archivos como si fuera root.

\subsubsection{}
El problema está en que \textit{gets} puede sobreescribir la dirección de
retorno si el string que se pasa tiene mas de 128 caracteres. Puede controlar
todo lo que este apilado debajo de la contraseña.

\subsection{}

\subsubsection{}
Si. Si lo que se le pasa a la funcion es un NaN.

\subsubsection{}
Podría devolver 0 en caso en que se le pase un NaN.

\setcounter{subsection}{7}
\subsection{}

\subsubsection{}
. ; cat /etc/passwd

\subsubsection{}
Puede pasarsele: .'' ; cat ''/etc/passwd

\subsubsection{}
Puede pasarsele: .'' \&\& cat ''/etc/passwd

\subsubsection{}
\begin{codesnippet}
\begin{verbatim}
void wrapper_ls(const char * dir) {
    pid_t pid = fork();
    if (pid == 0) {
        execve(``ls'', dir);
    } else {
        int status;
        while (waitpid(pid, &status, 0) == -1) {}
    }
}
\end{verbatim}
\end{codesnippet}