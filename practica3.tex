\section{Práctica 3}

\subsection{}

\subsubsection{}

No

\subsubsection{}

Puede imprimir 1 en los siguientes casos:

\begin{itemize}
    \item Si se ejecuta primero A y luego B
    \item Si colisionan las ejecuciones de el incremento de X, provocando que
    se ejecuten ambas líneas pero X se incremente sólo una vez
\end{itemize}

Puede imprimir 2 en caso de que se ejecuten B y luego A o que se ejecuten
simultaneamente pero que el printf sea lo último.

\subsection{}

Primero siempre imprime un 0 de la primer X, pero luego como Y ya es 1 puede
imprimir cualquier número de `a's, incluyendo 0, seguido de 1, 2, 3 y 4.

\subsection{}

Primero imprime `b' ya que A se queda haciendo busy waiting sobre X. Luego, es
posible que imprima ``adc'' o ``acd''.

\subsection{}

Si, los procesos acceden y modifican a la variable dentro del mutex, pero lo
liberan y luego toman decisiones respecto a ella, podría producirse una race
condition.

\subsection{}

Podría suceder que se produzca una inanición de los procesos que llaman
primero a wait() ya que quedan en el fondo de la pila hasta que todos los que
llegan después se liberen.

\subsection{}

Supongamos que wait() y signal() no se ejecutaran atómicamente. Supongamos que
tenemos una ejecución de dos procesos y ambos ejecutan wait(). Como wait() no
es atómica una ejecución posible es que ambos procesos lean la capacidad,
salgan del while, y al momento de decrementarla ya están los dos fuera del
ciclo, por lo que ambos ya son capaces de entrar a la zona crítica, violando
el principio de exclusión mutua.

\setcounter{subsection}{7}
\subsection{}

\begin{codesnippet}
\begin{verbatim}
bool TryToLock() {
    return !reg.testAndSet();
}
\end{verbatim}
\end{codesnippet}

\subsection{}

\begin{codesnippet}
\begin{verbatim}
Semaphore semaforos[N] = {0,...,0,1,0,...,0};

for (int j = 0; j < N; j++) {
    int pid = fork();
    if (pid == 0) {
        while (true) {
            semaforos[j].wait();
            P[j];
            semaforos[(j + 1) % N].signal();
        }
    }
}
\end{verbatim}
\end{codesnippet}

\subsection{}

\subsubsection{}

\begin{codesnippet}
\begin{verbatim}
// Semaforos para         A B C
Semaphore semaforos[3] = {1,0,0};

for (int i = 0; i < 3; i++) {
    int pid = fork();
    if (pid == 0) {
        while (true) {
            semaforos[i].wait();
            if(i == 0)
                A();
            else if (i == 1)
                B();
            else if (i == 2)
                C();
            semaforos[(i + 1) % 3].signal();
        }
    }
}
\end{verbatim}
\end{codesnippet}

\subsubsection{}

\begin{codesnippet}
\begin{verbatim}
// Semaforos para         B B C A
Semaphore semaforos[3] = {1,0,0,0};

for (int i = 0; i < 4; i++) {
    int pid = fork();
    if (pid == 0) {
        while (true) {
            semaforos[i].wait();
            if(i == 0 || i == 1)
                B();
            else if (i == 1)
                C();
            else if (i == 3)
                A();
            semaforos[(i + 1) % 4].signal();
        }
    }
}
\end{verbatim}
\end{codesnippet}

\subsubsection{}

\begin{codesnippet}
\begin{verbatim}
// Semaforos para          A  ByC
Semaphore semaforos[2] = { 1 , 0 };

for (int i = 0; i < 3; i++) {
    int pid = fork();
    if (pid == 0) {
        while (true) {
            if(i == 0) {
                semaforos[0].wait(2);
                A();
                semaforo[1].signal(2);
            }
            else if (i == 1) {
                semaforos[1].wait();
                B();
                semaforos[0].signal();
            }
            else if (i == 2) {
                semaforos[1].wait();
                C();
                semaforos[0].signal();
            }
        }
    }
}
\end{verbatim}
\end{codesnippet}

\subsubsection{}

\begin{codesnippet}
\begin{verbatim}
Semaphore sA = Semaphore(1);
Semaphore sB = Semaphore(0);
Semaphore sC = Semaphore(0);

bool sigueB = true;

for (int i = 0; i < 3; i++) {
    int pid = fork();
    if (pid == 0) {
        while (true) {
            if(i == 0) {
                sA.wait(2);
                A();
                if (sigueB) {
                    sigueB = false;
                    sB.signal(2);
                } else {
                    sigueB = true;
                    sc.signal();
                }
            }
            else if (i == 1) {
                sB.wait();
                B();
                sA.signal();
            }
            else if (i == 2) {
                sC.wait();
                C();
                sA.signal(2);
            }
        }
    }
}
\end{verbatim}
\end{codesnippet}

\subsection{}

Se utiliza un contador global. Después de cada $a_i$ se incrementa ese
contador y luego se hace un wait(), pero antes se pregunta si ese proceso es
el último en terminar. Si lo es se hace signal(N) para despertar a todos los
procesos que ya hayan terminado $a_i$. Es necesario utilizar un lock al
incrementar el contador para evitar una race condition. El lock debera abarcar
también la condición para evitar que dos procesos entren con el contador en
el mismo valor.

\begin{codesnippet}
\begin{verbatim}
a();
spin.lock();
cont++;
if (cont == N)
    sem.signal(N);
spin.unlock();
sem.wait();
b();
\end{verbatim}
\end{codesnippet}

\subsection{}

\subsubsection{}

No pueden terminar su ejecución.

\subsubsection{}

No entra en la definición ya que se encuentran esperando a que otros miembros
del grupo liberen el lock.

\subsubsection{}

No puede terminar su ejecución.

\subsubsection{}

Si, entra en la definición ya que se está a la espera de que se libere el
lock.

\subsection{}

\subsubsection{}

No se cumplen las condiciones de Coffman.

\subsection{}

Puede haber deadlock si un recurso toma $R_2$ y los otros dos toman $R_1$, y
todos necesitan ambos recursos para liberar el lock.

Si $R_1$ puede ser compartido por los 3 entonces no puede haber deadlock,
porque el que haya ganado el lock de $R_2$ puede liberarlo al obtener $R_1$, y
así con los 3 procesos.

\subsection{}

No puede haber deadlock ya que como los procesos necesitan a lo sumo dos
recursos, y son todos del mismo tipo, no hay forma de que alguno de los 3
procesos no consigo sus dos recursos, entonces si hay 2 procesos con un
recurso cada uno y el otro con 2, este los libera al terminar y ya los otros
pueden tomar uno cada uno.

\subsection{}

\subsubsection{}

Si, que cada proceso ejecute su primera linea, poniendo ambos semáforos en 0,
para luego atascarse en el wait del semáforo contrario.

\subsection{}

Se puede ver analizándolo proceso a proceso.

\begin{itemize}
\item $P_1$ y $P_3$ ya tienen lo que necesitan por lo que eventualmente
liberarán los recursos asignados: 2$R_2$, 3$R_1$ y 3$R_3$
\item Con eso $P_2$ puede cumplir sus necesidades (2$R_1$ y 2$R_3$) liberando
eventualmente 2$R_1$ y 1$R_4$ además de lo recientemente obtenido
\item Gracias a esto $P_4$ y $P_5$ ya pueden solicitar los recursos que
necesitaban y así terminar su ejecución, habiendo finalizado todos los
procesos
\end{itemize}

\subsubsection{}

Si, hay deadlock. Anteriormente ocurría que $P_1$ y $P_3$ liberarían los
recursos obtenidos ya que ya habían satisfacido sus necesidades, pero en este
caso $P_3$ todavía necesita 2$R_2$. El único que puede liberar recursos es
$P_1$, pero como libera únicamente 1$R_2$, esto no alcanza para satisfacer a
$P_3$ y por lo tanto $P_2$, $P_3$, $P_4$ y $P_5$ quedan a la espera de que se
les asigne sus respectivos recursos, sin liberarlos nunca.

\subsection{}

\subsubsection{}

\begin{codesnippet}
\begin{verbatim}
a();
spin.lock();
cont++;
if (cont == N)
    sem.signal(N);
spin.unlock();
sem.wait();
b();
\end{verbatim}
\end{codesnippet}

\subsubsection{}

\begin{codesnippet}
\begin{verbatim}
a();
atomReg.inc();
while(atomReg.get() != N) {}
b();
\end{verbatim}
\end{codesnippet}

\subsubsection{}

\begin{codesnippet}
\begin{verbatim}
a();
if (atomReg.getAndInc() == N - 1)
    sem.signal(N);
sem.wait();
b();
\end{verbatim}
\end{codesnippet}

\subsubsection{}

La más fácil de entender es en la que se usan sólo el registro atómico debido
a su simpleza.

\subsubsection{}

La más eficiente es la tercera ya que no desperdicia tiempo haciendo busy
waiting como las otras dos.

\subsection{}

\begin{codesnippet}
\begin{verbatim}
// Variables globales:
atomic<int> clientesDentro;

queue<Sempaphore cliente> colaLocal;
queue<Sempaphore cliente, Semaphore barbero> colaSofa;

TasLock lockLocal;
TasLock lockSofa;
TasLock lockPagando;

Semaphore quieroPagar, aceptoPago, pagoAceptado;
int esperandoAPagar;

// Cliente:

Semaphore yo = new Semaphore(0);
Semaphore barbero = new Semaphore(0);

// Si el local está lleno me voy al chori
if (clientesDentro.getAndInc() > 20) {
    clientesDentro.getAndDec();
    exit();
}

lockLocal.lock();
colaLocal.push(yo);
lockLocal.unlock();
entrar();

// Si hay espacio en el sofa voy y me siento
lockSofa.lock();
if (colaSofa.size() < 4) {
    lockLocal.lock();
    colaLocal.pop();
    colaSofa.push(<yo, barbero>);
    sentarseEnSofa();
    lockLocal.unlock();
    lockSofa.unlock();
} else {
    lockSofa.unlock();

    // Espero a que me llamen cuando se desocupe un lugar en el sofa
    yo.wait();

    // Me despertaron, me siento en el sofa
    lockSofa.lock();
    lockLocal.lock();
    colaLocal.pop();
    colaSofa.push(<yo, barbero>);
    sentarseEnSofa();
    lockLocal.unlock();
    lockSofa.unlock();
}

// Espero a que el barbero me llame
yo.wait();

\end{verbatim}
\end{codesnippet}

\begin{codesnippet}
\begin{verbatim}

lockSofa.lock();

// Si el sofa estaba lleno tengo que avisarle al proximo que se siente
if (colaSofa.size() == 3) {
    parado.signal();
}

lockSofa.unlock();

sentarmeEnSilla();
barbero.signal();

// Espero a que el barbero termine de cortarme el pelo
yo.wait();

// Se terminó el corte, voy a pagar
lockPagando.lock();
esperandoAPagar++;
lockPagando.unlock();
quieroPagar.wait();

// Pago
pagar();
aceptoPago.signal();
pagoAceptado.wait();

// Pago aceptado, salgo
clientesDentro.getAndDec();
salir();
exit();

// Barbero:

Semaphore yo, cliente;

while(true) {
    lockSofa.lock();
    if (!colaSofa.empty()) {
        <Semaphore,Sempahore> sentado = colaSofa.top();
        colaSofa.pop();

        cliente = sentado.cliente;
        yo = sentado.barbero;

        // Le aviso al cliente que venga a ser atendido
        cliente.signal();

        // Espero a que el cliente se siente en la silla
        lockSofa.unlock();
        yo.wait();

        // El cliente se sienta
        cortarCabello();

\end{verbatim}
\end{codesnippet}

\begin{codesnippet}
\begin{verbatim}

        //Le aviso que termine de cortar el pelo
        cliente.signal();
    } else
        lockSofa.unlock();

    lockPagando.lock();
    if (esperandoAPagar > 1) {
        esperandoAPagar--;
        lockPagando.unlock();

        // Le aviso a alguno que puede pagar
        quieroPagar.signal();

        // Espero a que me paguen
        aceptoPago.wait();

        // Me pagan
        aceptoPago();
        pagoAceptado.signal();
    } else
        lockPagando.unlock();
}
\end{verbatim}
\end{codesnippet}

\subsection{}

\begin{codesnippet}
\begin{verbatim}
Semaphore semMacho[N] = {0,...,0}
Sempahore semHembra[N] = {0,...,0};

void P(i, sexo) {
    if (sexo == macho) {
        semMacho[i].signal();
        semHembra[i].wait();
    } else {
        semHembra[i].signal();
        semMacho[i].wait();
    }
    entrar(i);
}
\end{verbatim}
\end{codesnippet}