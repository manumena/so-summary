\section{Práctica 5}

\subsection{}
Son necesarios $N$ accesos a memoria.

\subsection{}

\subsubsection{}
Es necesario leer los primeros 10 bloques directos.

\subsubsection{}
Es necesario leer los 12 bloques directos, el bloque de indirección simple y
los primeros 8 bloques del bloque de indirección. 21 en total.

\subsection{}

\subsubsection{}

\subsubsection{}
Conviene usar inodos ya que FAT no está acotado.

\subsubsection{}
Conviene usar FAT ya que mediante inodos esta acotado por la cantidad de
bloques del inodo.

\subsubsection{}
Conviene usar inodos porque se carga un inodo y los bloques del archivo a
medida que se van leyendo, en FAT se debe cargar la tabla entera a memoria.

\subsection{}

\subsubsection{}
Los identificadores de bloque son de 24 bits, por lo que es posible
direccionar hasta $2^{24}$ bloques. El del hash es de 16 bits por lo que el
tamaño de la tabla es de $2^{16}$ entradas.

La FAT contiene solo identificadores de bloque de 24 bits, por lo que su
tamaño es de $2^{24} \times 24$ bits. Lo que es lo mismo que
$2^{24} \times 3$ bytes $= 48MiB$.

En cuanto a la tabla de hash, cada entrada contiene el identificador de bloque
(24 bits) y el tamaño del archivo. El tamaño de un archivo no está actotado,
por lo que en principio podría llegar a ser de 16GiB, que es el tamaño del
disco. Por lo que para almacenar el tamaño del archivo se necesitan 34 bits,
siendo un total de 58 bits por entrada. El tamaño de la tabla de hash es de
$2^{16} \times 58$ bits $= 3801088$ bits $=14848$ bytes, poco más de 14KiB.

Como el tamaño de bloque es de dos sectores, y cada sector de 1KiB, la unidad
mínima es de 2KiB. La FAT queda almacenada a la perfección en 24Ki sectores,
pero el hash no, por lo que sufre de un poco de fragmentación interna; se
necesitan 7 bloques para cubrir 14KiB y un bloque extra para el resto.

Ambas estructuras ocupan entonces 24Ki + 8 bloques, lo que equivale a 50348032
bytes. Quedando mas de 15,95 Gib para los archivos.

\subsubsection{}
El tamaño del bloque debería ser de 2 sectores, ya que es lo que mejor se
ajusta al tamaño de los archivos, por mas que se desperdicie la mitad de la
memoria. Esto hace que la cantidad de bloques posibles a ser direccionados sea
de $\frac{16GiB}{2KiB} = 8Mi$ bloques $= 2^23$ bloques. Por lo que se
necesitan 23 bits para direccionarlos, lo cual genera la necesidad de que el
tamaño de los identificadores de bloque sea de 24 bits. Por último, como cada
archivo requerirá en promedio 1 bloque, puede decirse que se tiene la misma
cantidad de archivos que de bloques, entonces el tamaño del hash también es de
24 bits.

\subsubsection{}
En caso en que los archivos tuviesen un tamaño promedio de 16MiB, uno
intentaría usar el tamaño máximo de bloque que es de 8 sectores (8 KiB). Pero
la cantidad de bits a utilizar para los identificadores de bloque diferiría
en 2 (o en 1 si se usasen bloques de 4 sectores), lo cual lleva a utilizar la
configuración de 24 bits de todos modos, y lo mismo para el hash.

\subsection{}
Si, en caso en que varios procesos intenten leer el mismo bloque, con raid 0
deberán turnarse para leerlo, mientras que con raid 1 cada proceso puede leer
una copia distinta.

\subsection{}

\subsubsection{}
2 accesos: el bloque a escribir y el de paridad.

\subsubsection{}
9 accesos: 7 para los bloques, 1 para la paridad de los primeros 4 y otro para
la de los últimos 3.

\setcounter{subsection}{7}
\subsection{}

\subsubsection{}
Falso, siempre puede incrementarse el nivel de protección agregando una copia
más, en caso de que el resto de las copias fallen.

\subsubsection{}
La aumenta en el sentido de que esa cinta puede ser robada, pero también puede
ser robado el disco en donde está almacenada la información, y de hecho es más
vulnerable dentro del disco ya que es posible que sea accedido de manera
remota, lo cual no ocurre con la cinta.

\subsection{}
Un snapshot es una imágen del estado del sistema, una forma de recuperar