\section{Práctica 1}

\subsection{}

Se debe guardar el PCB y los registros del proceso desalojado, y luego cargar
el PCB y los registros del nuevos proceso asignado.

\subsection{}

\begin{codesnippet}
\begin{verbatim}
Ke_context_switch(PCB* pcb_0, PCB* pcb_1) {
    pcb_0->CPU_TIME += ke_current_user_time();
    pcb_0->STAT = KE_READY;
    pcb_0->PC = PC;
    pcb_0->R0 = R0;
    pcb_0->R1 = R1;
    pcb_0->R2 = R2;
    pcb_0->R3 = R3;
    pcb_0->R4 = R4;
    pcb_0->R5 = R5;
    pcb_0->R6 = R6;
    pcb_0->R7 = R7;
    pcb_0->R8 = R8;
    pcb_0->R9 = R9;
    pcb_0->R10 = R10;
    pcb_0->R11 = R011
    pcb_0->R12 = R12;
    pcb_0->R13 = R13;
    pcb_0->R14 = R14;
    pcb_0->R15 = R15;

    PC = pcb_1->PC
    R0 = pcb_1->R0
    R1 = pcb_1->R1
    R2 = pcb_1->R2
    R3 = pcb_1->R3
    R4 = pcb_1->R4
    R5 = pcb_1->R5
    R6 = pcb_1->R6
    R7 = pcb_1->R7
    R8 = pcb_1->R8
    R9 = pcb_1->R9
    R10 = pcb_1->R10
    R11 = pcb_1->R11
    R12 = pcb_1->R12
    R13 = pcb_1->R13
    R14 = pcb_1->R14
    R15 = pcb_1->R15

    pcb_1->STAT = KE_RUNNING;

    ke_reset_current_user_time();
    ret();
}
\end{verbatim}
\end{codesnippet}

\subsection{}

Una system call es un llamado a un servicio del kernel, como puede ser
realizar entrada/salida a un dispositivo o lanzar un proceso hijo mientras que
una llamada a función de biblioteca realiza operaciones con los registros y
direcciones de memoria en el espacio de direcciones del usuario.

\setcounter{subsection}{4}
\subsection{}

\begin{codesnippet}
\begin{verbatim}
pid_t homero = fork();
if (homero != 0) {
    wait_for_child(homero);
    printf('Abraham');
    exit();
} else {
    pid_t bart = fork();
    if (bart != 0) {
        pid_t lisa = fork();
        if (lisa != 0) {
            pid_t maggie = fork();
            if(maggie != 0) {
                wait_for_child(bart);
                wait_for_child(lisa);
                wait_for_child(maggie);
                printf('Homero');
                exit();
            } else {
                printf('Maggie');
                exit();
            }
        } else {
            printf('Lisa');
            exit();
        }
    } else {
        printf('Bart');
        exit();
    }
}
\end{verbatim}
\end{codesnippet}

\subsection{}

\begin{codesnippet}
\begin{verbatim}
void system(const char *arg) {
    pid_t pid = fork();
    if (pid == 0) exec(arg);
    else {
        wait_for_child(pid);
        exit();
    }
}
\end{verbatim}
\end{codesnippet}

\subsection{}

\subsubsection{}

Deben residir tiki y taka en el área de memoria compartida

\subsubsection{}

temp no debe residir en el espacio de memoria compartida

\subsubsection{}

\begin{codesnippet}
\begin{verbatim}
int main() {
    pid_t child = fork();
    if (child == 0) {
        taka_runner();
    } else {
        share_mem(&tiki);
        share_mem(&taka);
        tiki_taka();
    }
}
\end{verbatim}
\end{codesnippet}

\subsection{}

\subsubsection{}

\begin{codesnippet}
\begin{verbatim}
pid_t parent = get_current_pid();
pid_t child = fork();
int count = 0;
if (child == 0) {
    while (true) {
        breceive(parent);
        bsend(parent, 2 * count + 1);
        count++;
    }
} else {
    while (true) {
        bsend(child, 2 * count);
        breceive(child);
        count++;
    }
}
\end{verbatim}
\end{codesnippet}

\subsubsection{}

\begin{codesnippet}
\begin{verbatim}
pid_t parent = get_current_pid();
int count = 0;

pid_t child_1 = fork();
if (child_1 == 0) {
    while (true) {
        bsend(child_1, 3 * count + 1);
        breceive(child_2);
        count++;
    }
} else {

    pid_t child_2 = fork();
    if (child_2 == 0) {
        while (true) {
            breceive(child_1);
            bsend(parent, 3 * count + 2);
            count++;
        }
    } else {
        while (true) {
            bsend(child_1, 3 * count);
            count++;
            breceive(child_2);
        }
    }
}
\end{verbatim}
\end{codesnippet}

\subsection{}

\subsubsection{}

No es realizable ya que el proceso de la izquierda comienza enviando result,
por lo que hasta que el proceso de la derecha no reciba el mensaje, el de la
izquierda permanecerá bloqueado, ya que el sistema operativo posee una cola de
mensajes de capacidad cero.

Una ejecucion posible es que en el tiempo 1 se ejecute
$computo_muy_dificil_2$, y de ahi en adelante si pueden ejecutarse ambos.

\subsubsection{}

Podría utilizarse una cola de mensajes de mayor capacidad para evitar que el
proceso de la derecha se bloquee al enviar el mensaje.

\subsection{}

\subsubsection{}

Memoria compartida, porque es más rápida. En este caso las funciones no pueden
pisarse.

\subsubsection{}

Pasaje de mensajes. Si lo hiciera con memoria compartida tendría que haber
alguna especie de busy waiing para reconocer en que momento termina
cortarBordes. Con mensajes esto se hace sin desperdiciar recursos.

\subsubsection{}

Pasaje de mensajes. Es imposible usar memoria compartida ya que los procesos
se ejecutarán en computadoras distintas. Además, en caso de que sí fuera
posible, parece ser que por el nivel de seguridad al que está expuesto
eliminarOjosRojos, no es buena idea usar memoria compartida ya que podría ser
inseguro.

\setcounter{subsection}{11}
\subsection{}

Si sólo hubiera pipes como comunicación interprocesos el sistema se vería
enlentecido ya que la falta de señales haría imposible que un proceso fuera
interrumpido por lo que debería suplirse haciendo una especie de polling, lo
cual desperdiciaría mucho tiempo de ejecución.

\setcounter{subsection}{13}
\subsection{}

El error esta en utilizar $shared_child_pid$ siendo el padre para recibir del
hijo ya que es posible que el hijo todavia no se haya ejecutado, lo que
provocaría que no se le haya asignado un valor todavía a $shared_child_pid$.

La solución es reemplazarlo por child, que ya contiene el pid del hijo.

\setcounter{subsection}{15}
\subsection{}

\begin{codesnippet}
\begin{verbatim}
int assignPipe[2];
int resultPipe[2];

int file = open(``nombre_de_archivo'');
int *list;

pipe(assignPipe);
pipe(resultPipe);

for(int i = 0; i < 8; i++) {
    int child = fork();
    if (child == 0) {
        close(assignPipe[1]);
        close(resultPipe[0]);

        do {
            read(assignPipe[0], list, sizeof(list));
            write(resultPipe[1], calcular_promedio(list));
        } while(list != loquesea);

        close(assignPipe[0]);
        close(resultPipe[1]);
        exit();
    }
}

close(assignPipe[0]);
close(resultPipe[1]);

int rowsRead = 0;
while(cargarFila(file, list) || rowsRead < 8) {
    rowsRead++;
    write(assignPipe[1], list, sizeof(list));
}

char *proms;
int prom;
do {
    read(resultPipe[0], prom, sizeof(prom));
    proms ++ prom;
    
} while ()

close(``nombre_de_archivo'');
\end{verbatim}
\end{codesnippet}
