\section{Práctica 6}

\subsection{}
Polling sería más conveniente en caso de que se realice una comunicación
constante, para cualquier caso en que se envíe una ráfaga de datos es mucho
más costoso el tiempo perdido en cambios de contexto si se implementara el
uso de interrupciones que el tiempo de espera activa del polling.

Si la comunicación tuviese intervalos largos o no determinados, conviene la
estrategia de interrupciones, como en un teclado.

El caso híbrido es el que se envían ráfagas de datos separadas por intervalos
largos. Se utilizan las interrupciones para establecer la comunicación, se
envía la ráfaga de datos atendida mediante polling, y una vez finalizada se
espera a la próxima interrupción.

\subsection{}
El uso de E/S de un procesos en promedio es de 0.77, entonces el uso de la CPU
para 6 procesos es

\begin{center}
$U_{CPU}(6) = 1 - U_{E/S}(1)^6 = 1 - 0.2084 = 0.7916$
\end{center}

En cuanto al DMA,

\begin{center}
$U_{DMA}(6) = 1 - U_{CPU}(1)^6 = 1 - 0.23^6 = 0,9998$
\end{center}

\subsection{}

\subsubsection{}
La velocidad de los dispositivos virtuales va a ser la misma que la de los
procesos, no?

\subsubsection{}
La latencia mejora ya que los procesos no necesitan quedarse esperando a que
el dispositivo esté listo.

La liberación de recursos es mayor ya que no se hace espera activa.

El throughput mejora porque es posible usar los recursos liberados para otras
tareas.

\subsection{}

\subsubsection{}
Debe estar al tanto de que haya spooling.

\subsubsection{}
El driver no sabe que hay o no spooling, solo el usuario.

\subsection{}

\subsubsection{}
Si, el driver es código.

\subsubsection{}
No, el driver es solamente software.

\subsubsection{}
No. Es una porción de software hecha por el fabricante del dispositivo a
controlar.

\subsubsection{}
No, se ejecuta en espacio de kernel.

\subsubsection{}
Falso. Además de interrupciones, puede trabajar mediante polling o DMA.

\subsubsection{}
Verdadero?

\subsection{}
\begin{codesnippet}
\begin{verbatim}
int driver_init() {}
int driver_remove() {}
int driver_open() {}
int driver_close() {}

int driver_read(int *data) {
    *data = IN(CHRONO_CURRENT_TIME);
    return IO_OK;
}

int driver_write(int *data) {
    OUT(CHRONO_CTRL, CHRONO_RESET);
    return IO_OK;
}
\end{verbatim}
\end{codesnippet}

\subsection{}

\setcounter{subsection}{10}
\subsection{}

\subsubsection{}
\begin{codesnippet}
\begin{verbatim}
int write(int sector, void *data) {
    if (DOR_STATUS)
        OUT(DOR_IO, 1);
    sleep(50);

    OUT(ARM, sector / cantidad_sectores_por_pista());
    while(ARM_STATUS) {}

    OUT(SEEK_SECTOR, sector % cantidad_sectores_por_pista());

    escribir_datos(data);
    while(DATA_READY) {}
}
\end{verbatim}
\end{codesnippet}

\subsubsection{}
\begin{codesnippet}
\begin{verbatim}
Semaphore ready;
Semaphore timer;

void arm_or_data_ready() {
    ready.signal();
}

void timer_ready() {
    timer.signal();
}

int driver_init() {
    ready = new Sempaphore(0);
    timer = new Sempaphore(0);
    request_irq(6, arm_or_data_ready);
    request_irq(7, timer_ready);
}

int write(int sector, void *data) {
    if (DOR_STATUS)
        OUT(DOR_IO, 1);
    timer.wait();

    OUT(ARM, sector / cantidad_sectores_por_pista());
    ready.wait();

    OUT(SEEK_SECTOR, sector % cantidad_sectores_por_pista());

    escribir_datos(data);
    ready.wait();

    return(IO_OK);
}
\end{verbatim}
\end{codesnippet}